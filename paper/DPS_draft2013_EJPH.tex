% TODO do weighted lm(), do that the countries have the same weight?? 
% TODO redo police figure on aggregate level??

% Preamble  % {{{1
\documentclass[12pt]{article}
% Makes title and author left justified:
\makeatletter
\renewcommand{\maketitle}{\bgroup\setlength{\parindent}{0pt}
\begin{flushleft}
  {\LARGE \@title}

  {\large \@author}
\end{flushleft}\egroup
}
\makeatother
\usepackage{authblk}

\raggedright

\usepackage{abstract}
\renewcommand{\absnamepos}{empty}

% \usepackage[none]{hyphenat}
%\usepackage{achicago}
\usepackage[a4paper]{geometry}
\geometry{tmargin=3cm, bmargin=3cm, lmargin=3cm, rmargin=3cm}
\usepackage{setspace}
\doublespacing 

\usepackage[T1]{fontenc}
\usepackage[utf8]{inputenc}


% \setcounter{secnumdepth}{2}
\setcounter{tocdepth}{2}
\usepackage{placeins}                     
% \usepackage{appendix}
\usepackage{amsthm}
\usepackage{amsmath}
\usepackage{amssymb}
\usepackage{dcolumn}
\usepackage{makecell}
\usepackage{booktabs}
\usepackage{microtype}
% \usepackage[tt]{titlepic}
% \makeatletter
% \footnotemargin{-1em}
%\numberwithin{equation}{section}
%\numberwithin{figure}{section}
\clubpenalty=2000 
\widowpenalty=2000
\usepackage[flushleft]{threeparttable}
\usepackage{lscape}
\usepackage{amsbsy}
\usepackage{amsmath}
\usepackage{multirow}
\usepackage{graphicx}
\usepackage{longtable}
\usepackage{fixltx2e}
\usepackage{rotating}
\usepackage{float}
\usepackage{caption}
% \usepackage{floatrow}
\usepackage{varioref} % \vref{} gives also the page number.
\usepackage{tabularx}
\usepackage{pdfpages}
\usepackage{numprint}
\newcommand{\superscript}[1]{\ensuremath{^{\textrm{#1}} }}
\newcommand{\subscript}[1]{\ensuremath{_{\textrm{#1}} }}

% JELS formating:
\usepackage[hang, flushmargin]{footmisc} 
\renewcommand{\rmdefault}{ptm} % sets default font to Times New Roman

\usepackage{titlesec}
\titleformat{\section}{\large\bf}{\thesection.}{.5em}{}
\titleformat{\subsection}{\it}{\thesubsection.}{.5em}{}
\titleformat{\subsubsection}{\normalfont}{\thesubsubsection.}{.5em}{}

\usepackage[numbers,super,sort]{natbib}
\bibliographystyle{EJPH}

\captionsetup[figure]{labelsep=cq, labelfont=it}
\captionsetup[table]{labelsep=cq}
\DeclareCaptionLabelSeparator{cq}{:\quad}
\newcommand{\itemNoindent}{\item\hspace{-1mm}}
% end(JELS formating)

\begin{document}

% Titlepage  % {{{1
\title{A Radical Change in Traffic Law: \\Effects on Fatalities in the Czech
  Republic\thanks{I would like to thank to Daniele Bondonio, Brendan Dooley,
    Libor Dušek, Kateřina Holíková, Michael Kohl, participants at the 2010
    Conference on Empirical Legal Studies, the European Law \& Economics
    Association 2012 Annual Conference, and a workshop at the University of
    Economics, Prague for valuable comments and suggestions. I very much
    appreciate discussions with officers at the Czech Traffic Police
    Headquarters as well as their responses to data requests. The views
    expressed in this paper as well as any remaining mistakes and imperfections
    should be attributed only to the author.  This research was funded by Czech
    Science Foundation grant no. P402/12/2172.
  }
}

\author{\textsc{Josef Montag}}

\affil{\small CERGE--EI, Center for Economic Research and Graduate Education--Economics
  Institute, a joint workplace of Charles University in Prague and the Academy
  of Sciences of the Czech Republic} 

\affil{ \textbf{Correspondence:} Josef Montag,
  CERGE-EI, Politických vězňů 7, 111 21 Praha 1, Czech Republic, tel: +420 604
  715 714, e-mail: josef.montag@cerge-ei.cz}

\date{}

% \abstractcz{Studie hodnotí efekty Zákona o provozu na pozemních komunikacích v
%   České republice, který vstoupil v platnost 1. července 2006. Zákon přinesl
%   tvrdší postihy řidičů pomocí zavedení bodového systému a několikanásobného
%   zvýšení pokut, spolu s posílením pravomocí policistů ve výkonu služby. V
%   průběhu prvních tří měsíců platnosti zákona nastal prudký, 33,3 procentní,
%   pokles počtu smrtelných zranění. To představuje 51 až 204 zachráněných životů
%   s jistotou 95 procent.  Pokles byl však krátkodobý a odhady efektů
%   přesahujících první rok jsou kolem nuly.  Unikátní data o aktivitě dopravní
%   policie ukazují, že prostředky pro výkon služby postupně klesaly a byly více
%   alokovány na obecné vynucování práva.}

% \jelclass{I12, I18, K42, R41}

% \keywords{Traffic law, traffic fatalities, policy evaluation, deterrence,
%   enforcement.
% }

\maketitle

\renewcommand{\abstractname}{} % clear the title
\begin{abstract}
  \textbf{Background:} This study examines short- and long-run effects of a new
  road traffic law in the Czech Republic that became effective on July 1, 2006.
  The law introduced tougher punishments through the introduction of a demerit
  point system and a manifold increase in fines, together with an augmented
  authority of traffic police.  \textbf{Methods} Identification is based on
  difference-in-diferences methodology, whereas neighbouring countries serve as
  a control group. \textbf{Results:} I find a sharp, 33.3 percent, decrease in
  accident-related fatalities during the first three post-reform months. This
  translates into 51 to 204 saved lives with 95 percent certainty.  The decline
  was, however, temporary; estimates of the effects going beyond the first year
  are around zero. Unique data on traffic police activity reveal that police
  resources devoted to traffic law enforcement gradually declined and were
  shifted towards general law enforcement. \textbf{Conclusions:} The law had
  large, but shortlived effects.  Sustaining enforcement at pre-reform levels
  may be problematic and the decline in police resources may account for the
  absence of long-run effects of touger sanctions and the demerit point system.
\end{abstract}

\section*{Introduction} % {{{1
\label{sec:intro}

Each year, road traffic accidents (RTAs) result in as many as 50 million
injuries and more than 1.2 million deaths, making it the ninth leading cause of
death worldwide---effecting especially young people---and its importance is
predicted to rise over the next two decades.  Enacting comprehensive laws with
appropriate penalties and ensuring necessary resources for enforcement are
acknowledged as top instruments to improve road safety \citep{who_global_2009}.
However, proper policy choice requires that we understand how alternative
measures perform when put in place and how they interact with other key
variables. Exploiting past policy experiments is a natural way to improve our
understanding of these phenomena.

This study evaluates the effects of a new road traffic law in the Czech Republic
that became effective on July 1, 2006
\citep{parliamant_of_the_czech_republic_zkon_2005}. It was aimed at improving
road traffic safety through tougher sanctions for traffic offenses and the
augmented authority of the police. Apart from a manifold increase in fines, the
most important change introduced by the law was a demerit point system (DPS)
under which an accumulation of points for traffic offenses leads to the
suspension of driver's license. I overview the new road traffic law in detail in
Section \ref{sec:the-new-law}. 

There are over a dozen studies investigating the effects of similar changes in
traffic laws that also included a DPS recently adopted in other
countries.\footnote{Brazil did so in 1998, Ireland in 2002, Italy in 2003, Spain
  in 2006, and the United Arab Emirates in 2008. } I summarize these studies in
Table \ref{tab:BApapers} in the Appendix. The common pattern of their findings
is that the introduction of stricter traffic laws is followed by substantial
decreases in RTA-related fatalities and other casualties, usually in the realm
of 20 to 30 percent.\footnote{The study from the United Arab Emirates is an
  exception as it does not find any effects \citep{mehmood_evaluating_2010}.}
However, the effects going beyond the initial six months are ambiguous, as many
of these studies are based on short-term data and there are contradictions among
those that do look at long-run effects.  For instance, one study for Ireland
finds lasting effects, but two others do not.\footnote{See
  \citet{butler_trends_2006,healy_speed_2004,hussain_speeding_2006}.} Similarly
in the case of Italy, where one research group finds lasting effects and three
others do not.\footnote{See
  \citet{nicita_rational_2009,farchi_evaluation_2007,istat_instituto_nazionale_di_statistica_statistica_2005,zambon_evidence-based_2007,zambon_sustainability_2008}.}
Some of the inconsistency in previous findings may be related to research
design, which is always based on, a within-country, before-after comparison. One
should therefore be careful before drawing strong inferences, as such results
may be influenced by trends in the data and are fragile with respect to
additional shocks, such as seasonality, weather, change in fuel prices, or
business cycle.  

This study evaluates the effects of the Czech road traffic law reform using a
standard difference-in-differences set-up, whereas regions of neighboring
countries (Austria and Germany) serve as a control group. In short, I find a
very sharp but very short-lived reduction in fatalities. The paper adds three
main contributions to the existing literature: a) a better identification
strategy (previous evaluations of similar reforms in other countries used only a
before-after comparison), b) an explanation of the shortlived nature of the
effect by investigating how police enforcement activities responded to such a
radical change in penalties, and c) the finding that police enforcement
activities gradually declined after the reform, but the decline can hardly
explain the quick rebound in the number of deaths following the initial drop.

I have collected monthly regional-level data on RTAs that occurred between
January 2004 and December 2008 in the Czech Republic, Germany, and Austria and
matched it with other socio-economic and transport-related statistics. Because
data on accidents and injuries may suffer from reporting biases that are
correlated with the new traffic law, I focus on fatalities. To the extent that
the development of the variable of interest is similar across these countries,
the control group allows estimating the counterfactual, i.e. the hypothetical
scenario of what would have happened on Czech roads had the law not been
enacted. Subtracting the observed values from the counterfactual then yields an
estimate of the effect of the reform. The validity of the identifying
assumptions is discussed in Section \ref{sec:data-and-strategy}, but note here
that there is strong positive correlation in RTA-related fatalities across the
three countries, the three countries followed similar trends before the law was
introduced,\footnote{See Figure \vref{fig:fatalcar}.} and there was no major
change in Austrian or German traffic laws during the period under study.

Consistent with the experience from other countries I find a sharp---40.5 log
points---drop in fatalities during the first three months after the law became
effective. This translates into 51 to 204 saved lives with 95 percent certainty.
However, beyond the short-run impact, this paper extends the set of studies that
do not find lasting effects of increased sanctions for traffic law violations.
This result is robust to alternative specifications and controlling for GDP,
car-population ratio, age of cars, and freight-transport vehicle-kilometers.
Looking closer at the initial period, the effect was concentrated in July (the
point estimate is -83.3 log points). An analysis of daily data corroborates
these findings; moreover, there are no indices of pre-reform effects. The
strongest effects are found in the weeks immediately following the
enforceability of the reform.

So why were the effects short-lived? A possible concern is that the intensity
of enforcement decayed in the aftermath of an increase in punishment.
Intuitively, traffic law enforcement is costly and resources spent on it have
alternative uses, be it within law enforcement or within the public sector
in general. As the situation on the roads improves, alternatives may become more
attractive.
  
I find evidence consistent with this reasoning using a unique
monthly-regional-level dataset with detailed information on traffic police
activity during 2006 and 2007.  Specifically, while the number of traffic
policemen allocated to enforcement slightly increased, the total number of
man-hours in enforcement decreased by some 22 percent across the two years.  An
even faster decay is seen in the number of hours of the use of speed guns by
traffic police. On the other hand, the traffic police found more people at
large, more stolen vehicles, as well as conducted more vehicle and person
searches. The latter results, although rarely statistically significant, suggest
that some reallocation of resources may also have taken place within traffic
police itself. This may help explain the absence of longer-run effects.
However, continuous changes in police activity do not explain the initial sharp
drop in fatalities or the bouncing back. It is plausible that people simply
overestimated the effects of the change in the rules on the effective
punishments they faced. The salience of the change and lasting controversies in
politics and media may have contributed to this.

The remainder of this paper is organized as follows:  Section
\ref{sec:the-new-law} details the 2006 traffic law reform in the Czech Republic.
Section \ref{sec:data-and-strategy} describes the data and discuses the
empirical strategy.  Section \ref{sec:results} presents the main results.
Section \ref{sec:police} analyzes the behavior of the traffic police around the
reform and tries to figure out why the effects were shortlived.  Section
\ref{sec:conclusions} then concludes.

\section*{The Change in Czech Road Traffic Law} % {{{1
\label{sec:the-new-law}

The mechanics of the newly introduced DPS is straightforward and relatively
strict. The law newly specifies the number of demerit points for each offense,
from 1, for minor ones, to 7, for the most serious offenses. Drivers may receive
points for different offenses at one time. A driver who accumulates 12 points
has his license revoked for 12 months, automatically and
immediately.\footnote{The use the of masculine is justified by the fact that
  male drivers' behavior is more often associated with negative externalities
  than females drivers', both in conventional wisdom as well as empirically
  \citep{chipman_role_1975,steven_d._levitt_dangerous_2001,redelmeier_traffic-law_2003}.}
The license is returned to the driver upon completion of a driving test and the
driver continues to carry 12 points on his record. Four points are deleted from
driver's record after each 12 months during which he does not receive any new
points.\footnote{Before this change, a driver's license could have been revoked
  only upon conviction of a specific offense or crime.  It can still be revoked
  in such instances, regardless of the number of accumulated points and, in
  addition, the driver receives demerit points according to the offense he
  committed. If at the same time the driver happens to exceed his 12-point
  limit, the 12-month period begins only after the main revocation period is
  completed.} 
  
The toughness of the Czech DPS can be highlighted by comparison with the Italian
one, where drivers must only attend a driving course and pass a driving test
within 30 days after exhausting their point endowment; they also get all their
points back.\footnote{Drivers in Italy have an endowment of 20 points and they
  lose points upon committing an offense.} The license is only suspended if
they fail to pass the test.\footnote{\citet{nicita_warning_2012} indeed find
  that drivers tend to commit more offenses after zeroing out and reloading
  their points.}

The introduction of DPS was complemented by a general increase in fines and
police officers' discretion as to the actual amount of fine was removed in most
cases.  Maximum fines for offenses that can be solved on the spot, if driver
accepts the ticket, were mostly raised twofold, but fines for speeding were
tripled, as illustrated in Figure \ref{fig:fines}.\footnote{Median wage before
  taxes was 19,500 Czech Crowns (CZK) in 2006. One Euro equals approximately 25
  CZK, one U.S. Dollar is about 19 CZK.} Similarly, fines were increased for 
more serious offenses that are dealt with by the municipal office.  Driving
under the heavy influence of alcohol became a jailable crime.

% FIGURE \ref{fig:fines} ABOUT HERE
\begin{figure}[t]
  \caption{Average fine for speeding in 2006 and 2007}
  \label{fig:fines}
  \makebox[\textwidth]{ 
    \includegraphics[scale = .65]{../tables_figures/fig-fine_per_radar.pdf} 
  }
\end{figure}

Other key changes the law introduced are summarized in Table
\ref{tab:provisions}. Notably, the authority of the police was elevated
substantially as the law sought to strengthen enforcement. Police regained the
capacity to retain a driver's license on the spot and if a driver refuses an
alcohol test, the police can seize his vehicle or prevent the driver from
continuing. Municipal police were newly awarded the authority to stop vehicles,
impose fines, and give alcohol tests.  Radar detectors became illegal, while
child seats and all-day lighting were made compulsory. Vehicle owners became
obliged to provide information on the identity of the driver in order to make
the offenses documented by static speed cameras better enforceable. 

All in all, the law can be plausibly described as having substantially altered
the formal rules that govern road traffic in the Czech Republic. The rules of
the newly introduced demerit point system are strict---only two or three
offenses can add up to 12 points, resulting in an immediate year-long license
suspension---and this was accompanied by a general increase in fines and
augmented police authority. 

\section*{Data and Empirical Strategy} % {{{1
\label{sec:data-and-strategy}

\subsection{Descriptive Statistics and Measurement Issues} % {{{2
\label{subsec:descriptive-statistics}

\subsubsection{Data Collection} % {{{3 

The main data set analyzed in this paper consists of monthly regional-level data
on RTA-related casualties that occurred between January 2004 and December 2008
in the Czech Republic, Germany, and Austria, obtained upon specific requests
from the Czech Traffic Police Headquarters and statistical offices of Germany
and Austria.\footnote{I have also made data requests to the Polish, Slovakian,
  and Hungarian statistical offices, however I was not successful in those
  cases.} I then merged this data with yearly regional-level data on the
population and number of cars from Eurostat and yearly country-level data on
transport and economic statistics from the same source. In addition, I received
daily data on fatalities in Austria and the Czech Republic covering 2005 to
2008.  Finally, from the Czech Traffic Police Headquarters I have obtained
detailed information on traffic police activity, such as man-hours, hours of use
of speed guns, the number of cleared offenses, and the amount of collected
fines. From this information I was able to parse a regional-level dataset
covering monthly police activity in the Czech Republic in 2006 and
2007.\footnote{The data on traffic police activity is described and analyzed in
  Section \ref{sec:police}.}

\subsubsection{Fatalities and Injuries} % {{{3

Table \ref{tab:summary1} summarizes the data on RTA-related casualties in
Austria, the Czech Republic, and Germany (split by former East and West) before
and after July 1, 2006. Looking at the levels of fatalities, the Czech Republic
had the highest rate per million inhabitants as well as per million cars in both
periods. From the first column it is also apparent that the two neighboring
countries experienced a decline in the number of RTA-related fatalities, which
was comparable---with the exception of the former East German regions---to the
decline in fatalities in the Czech Republic. However, the number of cars
cruising Czech roads grew 3 to 4 times faster compared to Austria and Germany,
as seen in the last column. If the number of cars reflects the intensity of
traffic in a country, the rate of fatalities per car is more likely to capture
the safety situation on the roads. This adjustment leads to a 14 percent decline
in fatalities in the Czech Republic and former East Germany compared to 8 and 9
percent decreases in Austria and Germany, respectively.  

The picture is quite different when we look at injuries, however. First, the
Czech Republic exhibits the lowest injury rates for both serious and slight
injuries. The difference is substantial: Czech rates of RTA-related slight and
serious injuries per million inhabitants are about 50 percent lower compared to
Austria and Germany, while the number of injuries per million cars is still
about 1/3 smaller. This looks to be somewhat at odds with the  larger fatality
rates in the Czech Republic relative to its neighbors.  Regarding changes over
time, we see that---despite the declines in fatalities---injuries per million
inhabitants in Austria and Germany remained relatively stable; only serious
injuries declined by about 4 percent in former East Germany.  Similarly,
injuries per million cars declined rather modestly in these two countries. In
all cases, the decline in injuries is much smaller than the decline in
fatalities.  However in the case of the Czech Republic, the decline in injuries
always exceeds the decline in fatalities, most notably for the seriously
injured. 

There are two sources of concerns about the comparability of police data on
injuries across the three countries as well as over time. First, police
resources are likely to differ across these countries and drivers have
incentives to avoid calling the police to an accident involving injury in order
to sidestep additional punishment. In marginal cases, drivers may also strike a
deal and settle the damages privately. As a consequence, there may be
differences in the share of accidents the police never learn about.

% TABLE \ref{tab:summary1} ABOUT HERE
\begin{table}[t] 
  \scriptsize  
  \makebox[\textwidth][c]{
    \begin{threeparttable} 
      \caption{Summary Statistics of RTA Related Casualties (July 2004--June
        2008)
      }  
      \label{tab:summary1}   
      \begin{tabular}{
          l c *{6}{m{1cm}<{\raggedleft}} m{1.4cm}<{\centering}
        }  
        \toprule 
        \input{../tables_figures/tab-summary1.csv}
      \end{tabular} 
      \begin{tablenotes} \scriptsize 
      \item \hspace{-1.8mm} \textsc{Notes}: The first and last 6 months were
        dropped to make the within-country comparison free of seasonal effects.
        The number of fatalities for the Czech Republic refer to people who died
        within 24 hours of an accident, for Austria and Germany to people who
        died within 30 days of the accident. The estimate of the number of cars
        in Germany was revised downward in 2007; since this variable was
        developing virtually linearly, I replaced the revised numbers with the
        linear extrapolation.
      \item \hspace{-1.8mm} \textsc{Sources}: Headquarters of the Police of the
        Czech Republic, Statistics Austria, Federal Statistical Office Germany,
        and Eurostat. 
      \end{tablenotes} 
    \end{threeparttable} 
  }
\end{table} 

Second and more importantly, it is quite likely that these reporting issues have
been aggravated by the changes in the Czech road traffic law because the
incentives not to call the police strengthen as the punishments became harsher.
In addition, the law made such strategies cheaper, because it raised the minimum
damage below which the police do not have to be notified about an accident
from CZK 20,000 to CZK 50,000.\footnote{Although this change was motivated by
  the need to free police resources from dealing with minor accidents, it may
  also work against the general philosophy of the new law, because offenders who
  cause minor accidents go unpunished. I am grateful to Lt.  Col. Josef
  Tesa\v{r}\'{i}k for pointing this out to me. Another ramification for our
  study is that police data on accidents are not suitable for the evaluation.}
Both factors are likely to increase the pool of accidents and injuries unnoticed
by the police. At the same time, the police or the doctors may have an incentive
not to record some injuries, or classify injuries on the margin as minor,
if the related punishment would now seem inappropriate. The importance of these
factors possibly increases with the corruptibility of the police.

This scenario seems to fit the development of RTA-related injuries in the Czech
Republic and the inconsistency in dynamics when compared with Austria and
Germany.  It is also consistent with the larger decline in serious injuries
compared to slight injuries (see Table \ref{tab:summary1}) as some injuries on
the margin may be more often classified as slight and some slight injuries get
concealed. Since there are no marginal fatalities, the effect of classification
should be stronger in the case of serious injuries than slight injuries.

Such reporting issues are unlikely to play a role in the case of fatalities.
First, it is hard to conceal an accident involving a fatality from the police.
Second, as just noted, there are no marginal fatalities that could be labeled
otherwise.  Third, any private settlement is hardly feasible.\footnote{It is
  possible that some RTA-related fatalities may be labeled as not related to
  RTA. I consider this possible effect of the law unimportant.} 
                            
To probe things further, I compared the police data with yearly data on road
traffic fatalities and injuries from the  Institute of Health Information and
Statistics of the Czech Republic (IHIS). With the exception of Prague, the IHIS
data on fatalities were very similar to the police
data.\footnote{\label{fn:uzis-data}The proportional differences between the IHIS
  and the police statistics each year were  $-0.048$, $-0.019$, 0.023, 0.021,
  and 0.085 in 2004 to 2008.  If the doctors' data were correct, the police data
  slightly overstate the number of fatalities in the first two years in our main
  sample and understate it from 2006 onwards, notably in 2006, biasing results
  in this study in favor of finding negative effects of the change in law on
  fatalities.  Nevertheless figures from both sources are comparable both in
  level as well as their behavior over time.} The comparison looks different for
injuries. While the police observed a sharp, 14.6 percent, decline in
RTA-related injuries in 2006, the IHIS figure remained essentially unaltered (it
was slightly higher in fact).  The following year the police figure rose by
about 4 percent while the IHIS figure was about 11 percent down from the
previous year.\footnote{The proportional differences between the IHIS and the
  police statistics each year were, 0.13, 0.24, 0.44, 0.24, and 0.18 in 2004 to
  2008.} These comparisons support the claim that the police data on RTA-related
injuries are problematic. For these reasons I focus solely on data on fatalities
in what follows.       

This analysis also resulted in dropping Prague from the analysis. Prague
constituted an outlier with very different behavior from the rest of the
country. In addition, there were discrepancies between the police data and data
from the Institute of Health Information and Statistics.

\subsubsection{Transport Statistics and the Economy} % {{{3 

Table \ref{tab:summary2} presents a summary of the available transport
statistics and GDP from Eurostat. Compared to Austria, the kilometers driven by
Czech cars increased substantially. At the same time, the average number of
passengers per car was decreasing, in fact canceling out the increase in
kilometers driven and resulting in a slight decrease in passenger-km per car in
the Czech Republic over the period under study. 

The high intensity of transport traffic after the Czech Republic became a member
of the European Union in 2004 is often mentioned as potentially elevating the
riskiness of Czech roads. During the years 2004--2008, the volume of freight
transport (including empty truck movements) in the Czech Republic increased
substantially, but it also did in Germany.

Finally, the second half of the 2000s was an era of rapid economic growth in the
Czech Republic---real GDP per capita measured in 2005 Euros increased by 32
percent.\footnote{This partially reflects the strengthening of the Czech Crown
  throughout the period, however the growth in Crowns would still be at 20
  percent.} The expectation of the effect of GDP growth on RTA-related
fatalities per car is ambiguous; in the short run, one would expect increased
traffic to dominate, implying a positive correlation between GDP and fatalities.
In the medium and longer perspective, people may acquire better cars as their
incomes increase, families may buy a second car, and also public infrastructure
may improve, resulting in a negative effect on fatalities.

\subsection{Empirical Strategy} % {{{2
\label{subsec:estimator}

\subsubsection{The Outcome of Interest} % {{{3

As we saw in Table \ref{tab:summary1}, the number of cars grew much faster in
the Czech Republic compared to Austria and Germany. Assuming cars are bought to
be driven, the number of cars should be positively related to the total number
of kilometers driven in a country.  This suggests that the intensity of road
traffic was changing and grew faster in the Czech Republic relative to the two
neighbors. Under such circumstances, it is not feasible to take the number of
accidents or fatalities in a country as a measure of traffic safety, as risk
exposure changed.\footnote{For instance, if people decided to double the number
  of kilometers driven per day, it would not be surprising to see an increase in
  the number of accidents, while it would be difficult to make any statement as
  to the implied change in traffic safety.}  I therefore use the number of
fatalities per one million cars as a measure of road safety and the variable of
interest.

An issue with using the number of cars to adjust for changes in traffic
intensity can be that the number of kilometers per car or the average number of
passengers sitting in a car may change. Both variables are very likely
positively related to fatalities.\footnote{Other things remaining equal, more
  kilometers driven imply more opportunities for accidents to happen, while more
  passengers in a car imply more injuries or fatalities in any given accident.}
According to the statistics from Eurostat (Table \ref{tab:summary2}),
passenger-kilometers per car declined by almost 5 percent in the Czech Republic
(kilometers driven grew by 20 percent, but the decrease in the number of
passengers per car offset it) and did not change in Austria between 2004 and
2008.\footnote{Data for Germany is not available.} This indicates that using the
number of cars as a measure of exposure to the risk of dying in a traffic
accident is biased upwards in the case of the Czech Republic after 2006; this
implies a downward bias of estimates of the effect of the legal change on
fatalities, i.e.  making it more likely that a---long run---negative effect of
the law on fatalities will be found.\footnote{This reasoning abstracts from the
  indirect effect of a higher number of car-kilometers on the probability of an
  accident working simply through more cars meeting on the road.  Also, new
  drivers may be more likely to cause an accident, a factor possibly at work in
  the Czech Republic with its rapidly growing stock of cars.  (On the other hand
  new cars are safer.) I suggest these factors are likely to be dominated by the
  mechanics of kilometers driven and passengers sitting in a car.}

\subsubsection{Empirical Model} % {{{3

This paper employs a difference-in-differences (DD) estimator using neighboring
countries as a control group. Specifically, denote $y$ the outcome of interest,
then a DD regression can be written as \begin{equation} \label{eq:DD} y_{rt} =
  \rho_{r} + \boldsymbol{\tau_{t}} + \boldsymbol{\beta'}C_c\boldsymbol{T_{t}} +
  \boldsymbol{\gamma'}\boldsymbol{x_{rt}} + \epsilon_{rt}, \end{equation} where
$c$ denotes a country; $r$ denotes a region; $t$ denotes the date; $\rho_{r}$ is
region $r$'s fixed effect; $\boldsymbol{\tau_{t}}$ is a full set of time effects
picking up common trends, shocks, and seasonal regularities; $C_c$ is a dummy
for the Czech Republic; $\boldsymbol{T_t}$ is a vector of time effects;
$\boldsymbol{x_{rt}}$ is a vector of additional controls; and $\epsilon_{rt}$ is
the residual.  Specifically, $\boldsymbol{T_t}$ consists of dummies for four
post-reform quarters and a dummy for the fifth to tenth quarter after the
reform.  The parameter of interest is the vector $\boldsymbol{\beta}$, which
consists of five coefficients capturing the effects of the traffic law reform on
fatalities over time.

The well-known advantage of this estimator is that it helps with controlling for
any unobserved shocks, as long as they affect the treated as well as the control
group. This facilitates more robust inference about the casual effect of the
change in road traffic law on RTA-related casualties relative to before-after
comparisons employed in existing studies. The identifying assumptions of the DD
estimator are common trends between the treated and the control group as well as
the absence of any unobserved shock specific only to the control countries or to
the treated country. In other words, the DD estimator requires that after
controlling for relevant differences between the control and treated groups, the
only systematic difference between the two is the presence of the treatment. 

\subsubsection{Validity of Assumptions} % {{{3 

There are good reasons to expect that factors generating shocks to RTA-related
fatalities are shared among the Czech Republic, Austria, and Germany, so the
neighboring countries offer themselves as natural control group. Specifically,
common borders and the relatively small size of the Czech Republic make it
likely that weather conditions will affect these three countries similarly.
Moreover, the Czech economy is export-oriented---exports represented over 60
percent of its GDP in 2006, while Germany is the main trading partner with a
$1/3$ share of Czech exports.  Thus, economic shocks are correlated across these
three countries.  Finally, there was no substantial policy change in either
control country.\footnote{I have sent inquiries about changes in traffic law or
  policies between 2004 and 2008 to the ministries of transport of Austria and
  Germany. A detailed answer came from the Austrian ministry, listing all
  changes that occurred in that period. These were limited provisions such as an
  obligation to carry a reflective vest (2005) or the obligatory use of winter
  tires (2006, the only change in that year) or an increase of the fine for
  using a cell phone in 2008. I did not receive any response from the German
  ministry, however researching publicly available resources did not result in
  finding any substantial law or policy change in Germany between 2004 and
  2008.} 

% FIGURE \ref{fig:fatalcar} ABOUT HERE
\begin{figure}[t]
  \caption{Fatalities per $10^6$ Cars (2004--2008)} 
  \label{fig:fatalcar}
  \makebox[\textwidth]{ 
    \includegraphics[scale = .65]{../tables_figures/fig-fatalities.pdf} 
  }
\end{figure}

Since our data span 30 months before the traffic law reform in the Czech
Republic, we can use this information to probe the identifying assumptions of
regression (\ref{eq:DD}). Figure \ref{fig:fatalcar} provides visual evidence
suggesting similar log-linear trends in fatalities per car before 2006 as well
as a similarity of seasonal regularities across the Czech Republic, Austria, and
former West Germany.\footnote{\label{fn:DDR}Because the regions of former East
  Germany exhibited a very different trend compared to the rest of the countries
  in the region---namely, fatalities were falling much faster in Eastern Germany
  than elsewhere---I drop them from the data. As discussed below, I also studied
  yearly regional-level data from Eurostat, which are available for all
  neighboring countries. The uniqueness of former East Germany applies not only
  with respect to the three countries present in our main data, but to two other
  neighboring countries, Poland and Slovakia, as well as to Hungary.}  Pearson's
correlation coefficients of log monthly fatalities per million cars between the
Czech Republic and either control country from January 2004 until June 2006 are
above 0.7 and are statistically significant at any conceivable level.

To assess the pre-reform similarity between the Czech Republic and Austria and
Germany more formally, I construct a Chow test of systematic deviations of
fatalities in the Czech Republic from the control group. Specifically, I take a
pre-July 2006 subset of the data and run a regression of log-fatalities per car
on regional fixed effects, a full set of time effects (i.e. year-month dummies),
and a set of interactions between time effects and a dummy for the Czech
Republic. Time effects in this regression pick up trends and shocks common to
all three countries, while the interactions capture deviations specific to the
Czech Republic. The test of the hypothesis that all coefficients on the
interaction terms are equal to zero produces $F(29, 754) = 0.71$, in other
words, there is no evidence that fatalities before July 2006 behaved differently
in the Czech Republic compared to Austria and Germany.

To double-check this contention, I run the same test using the Eurostat yearly
regional-level data on fatalities between 1999 and 2005 for all neighboring
countries, replacing year-month dummies with year dummies. This data include
Austria, the Czech Republic, former West Germany, Poland, and
Slovakia.\footnote{While data are available from 1996, there are frequent
  revisions, especially in statistics on number of cars, making the earlier data
  less reliable. Also, the Czech Republic experienced a financial crisis in
  1997, recovering in 1999. I therefore drop observations before 1999. As
  discussed in Footnote \ref{fn:DDR}, I excluded the regions of former East
  Germany, since it exhibits a very different trend from all the other
  countries.}  Testing the hypothesis that year effects for the Czech Republic
are equal to zero results in $F(6, 228) = 0.889$.  Running the same test on data
for that only include the Czech Republic, former West Germany, and Austria
produces $F(6, 156) = 0.7947$.

These findings suggest strong similarities in the development of fatalities
between the Czech Republic and Austria and Germany before July 2006. That in
turn justifies the use of the difference-in-differences estimation strategy.

\subsubsection{Meaning of Estimates} % {{{3 

In our case, the coefficients $\boldsymbol{\beta}$ from regression \ref{eq:DD}
represent time-varying treatment effects on the treated, as the policy change
was designed and introduced by the Czech Republic itself, unlike in a random
assignment. Thus, $\beta$s evaluate the effects of the legal reform in the Czech
Republic and cannot be thought as an estimate of the effect in an experimental
sense.\footnote{This self-selection should make it more likely for the treatment
  to work compared to a randomly assigned treatment.  Also, as apparent from
  Table \ref{tab:summary1}, Czech roads were more dangerous than German and
  Austrian roads, the likely benchmarks, and there was no sign of convergence in
  fatalities per capita before 2006. Thus, some reform might have been be
  thought necessary.} There can also be a potential endogeneity bias in the
sense that the timing of the reform may not be random. A recent experience of an
unusually high number of serious accidents may make it more likely that a policy
change will be put in place. This would constitute a bias in the direction of
finding an effect due to the regression fallacy, as a random spike in fatalities
is likely to be followed by values closer to the average. However, the length of
the legislative process and the time span between the passage of a law and the
date from which it is effective---9 months in the case we study---make this
factor unlikely to drive our results in an important way.\footnote{The traffic
  law was passed on September 21, 2005 and became enforceable on July 1, 2006.}
Importantly, these biases work against the finding of no significant effect of
the reform and therefore make the conclusions of this paper conservative. 

The last potential bias is related to the pre-reform effects of the new road
traffic law.  Everyone was long aware that the change in the road traffic law
was coming soon and this may have affected pre-reform outcomes, in fact
generating some effects of the reform before it actually took
place.\footnote{For instance, police may have acquired new assets or invested in
  new technologies before the reform, or they may have strategically increased
  their effort prior the reform, or engaged in saber rattling in the media.
  Also, drivers may have begun to drive more carefully or pay more attention to
  road signs. They may also have expected the police to increase their effort
  around the introduction of the reform.} This would be likely to create a
downward bias in $\beta$s as the level of fatalities in the treated country
prior the reform would be lower. The potential pre-reform decrease should be
attributed to the reform itself, since it would have not occurred otherwise.
This bias is likely to be less dramatic the longer the data for the pre-reform
period.  There are also simple strategies to address this: one can include a
dummy for some part of the period prior the reform and interact it with the
dummies of the treated group. I also inspect the daily data on RTA-related
fatalities and find no appreciable pre-reform effects.

\section{Results} % {{{1
\label{sec:results}

\subsection{Main Results} % {{{2

The main set of ordinary least squares estimates of the effects of the new Czech
road traffic law on fatalities using regression (\ref{eq:DD}) are reported in
Table \ref{tab:DDbase}.\footnote{Because there are some month-region
  observations where no accidents occurred and I code them as zero, OLS
  coefficients are biased due to censoring. I rerun the analysis using a Tobit
  model instead. The differences in coefficients were barely discernible.
  Dropping those observations did not change the results, either. I therefore
  prefer to report OLS results.} Specification (1) (I will refer to it as the
``base result'') shows that the immediate effects of the law were substantial,
but shortlived. The point estimate for the first post-reform quarter suggests
that fatalities in the Czech Republic declined by one third ($ = [e^{-0.406} -
1] \times 100$).\footnote{Most of this decline was actually concentrated in
  July, the first month the law was in place, when fatalities fell by 55
  percent. I do not report this regression to save space.} However, the
estimated effects during the second, third, and fourth post-reform quarters are
comparatively smaller and never statistically significant, individually or
jointly.\footnote{Testing whether these three interactions are jointly equal to
  zero yields $F(3, 1473) = 0.45$.} The estimate of the last coefficient
suggests that the long-run effects, that is, beyond the first year after the
traffic law reform, were essentially zero. 

% TABLE \ref{tab:DDbase} ABOUT HERE
\begin{table}[t] 
  \footnotesize  
  \makebox[\textwidth]{ 
    \begin{threeparttable}
      \caption{Effects of the New Traffic Law on Fatalities} 
      \label{tab:DDbase} 
      \input{../tables_figures/DDresults.tex}
      \begin{tablenotes} \scriptsize 
      \item \hspace{-1.8mm} \textsc{Notes}: The outcome variable in
        specifications (1), (2), and (4) is the monthly log of fatalities per
        $10^6$ cars in Austrian, Czech, and German regions between 2004 and
        2008. Specification (2) is run on a sample without the distant regions
        of Austria and Germany. The outcome variable in specification (3)  is
        log fatalities per $10^{11}$ passenger-kilometers---data available only
        for Austria and the Czech Republic. All specifications include region
        dummies and an unrestricted set of month$\times$year effects.
        Huber-White standard errors clustered on regions are in parentheses: *
        $p<0.01$.
      \item \hspace{-1.8mm} \textsc{Sources}: Headquarters of the Police of the
        Czech Republic, Statistics Austria, Federal Statistical Office Germany,
        and Eurostat.
      \end{tablenotes} 
    \end{threeparttable}
  } 
\end{table}

The rest of Table \ref{tab:DDbase} reports alternative specifications to check
the robustness of this base result. In specification (2) I drop the distant
regions of Austria and Germany. The idea is that neighboring regions may be more
alike in their behavior over time as well as affected by similar factors. This
results in somewhat smaller and shorter estimated effects, yet the big picture
is not altered. Then, in specification (3), I replace the outcome variable, log
of fatalities per car, by the log of fatalities per passenger-kilometer, which
is a preferred measure of traffic intensity. Because the data on
passenger-kilometers are not available for Germany, the data includes only the
Czech Republic and Austria. The main findings  are again corroborated, although
the initial effects seem to decay even faster---everything beyond the initial 6
months has positive point estimates. Lastly, to check for potential pre-reform
effects, I added dummies for two pre-reform quarters in specification (4).  Both
pre-reform coefficients are close to zero and the estimated effects are
therefore the same as in specification (1).\footnote{I also estimated a model
  with the interaction of a dummy for the first quarters with a dummy for the
  Czech Republic, to control for different seasonal patterns. The results did
  not appreciably differ.} To summarize, these findings provide strong evidence
of substantial immediate effects of the new road traffic law on fatalities in
the Czech Republic, but not much is apparent beyond that.

% TABLE \ref{tab:DDcontrols} ABOUT HERE
\begin{table}[t] 
  \footnotesize  
  \makebox[\textwidth]{
    \begin{threeparttable}
      \caption{Effects of the New Traffic Law: Controlling for GDP and Transport
        Variables
      } 
      \label{tab:DDcontrols} 
      \input{../tables_figures/DDcontrols}
      \begin{tablenotes} \scriptsize 
      \item \hspace{-1.8mm} \textsc{Notes}: The outcome variable is the log of
        fatalities per $10^6$ cars. Specification (6) is run on a sample without
        the distant regions of Austria and Germany. All specifications include
        region dummies and an unrestricted set of month$\times$year effects.
        Huber-White standard errors clustered on regions are in parentheses: *
        $p<0.01$.
      \item \hspace{-1.8mm} \textsc{Sources}: Headquarters of the Police of the
        Czech Republic, Statistics Austria, Federal Statistical Office Germany,
        and Eurostat. 
      \end{tablenotes} 
    \end{threeparttable}
  } 
\end{table}

Table \ref{tab:DDcontrols} probes our base results with additional control
variables. I first include GDP per capita, which grew faster in the Czech
Republic compared to Austria and Germany, as seen in the last column of Table
\ref{tab:summary2}. Then I plug in the number of kilometers driven by trucks and
lorries. The increase in freight transport after May 2004 when the Czech
Republic became a member of the EU,\footnote{It did so on May 1, 2004.} was
often criticized by media and politicians as adversely affecting the safety of
Czech roads. Also, the number of cars per capita grew faster in the Czech
Republic than in Austria and Germany.  This may negatively influence the number
of fatalities per car, because the number of passengers sitting in a car may
decrease (see Table \ref{tab:summary1}) and the new cars may be safer. Lastly
the age of cars may capture changes in the composition of the quality of cars.
The signs of coefficient estimates are as expected, except freight transport
does not seem to positively affect fatalities, but none of these variables is
statistically significant on its own. Importantly, the initial effects of the
new traffic law remain highly statistically significant and substantively large
in all four cases, also coefficients for the second to fourth post-reform
quarters are quite stable, but never significant. However, the estimated effects
beyond the first year are somewhat unsteady.

In specifications (5) and (6) I include all control variables simultaneously,
whereas specification (6) is run on the restricted sample. Coefficients on
control variables generally have the same signs and magnitudes across all six
specifications. Notably the effects estimated in these last two models are
virtually the same as in corresponding specifications (4) and (5) in Table
\ref{tab:DDbase}. The base results are corroborated by the results in Table (6).

In summary, our results show that there were substantial short-run effects
concentrated within the first quarter after the law became effective. This is
consistent with the findings of previous studies in other countries (see Table
\ref{tab:BApapers} for a summary of these papers). Although the point estimates
are mostly negative, the estimated effects fade as we move away from July 2006.
The estimates of long-run effects, that is beyond the first 12 post-reform
months, are substantively and statistically insignificant.   

\subsection{Short-Run Development} % {{{2

The availability of daily country-level data on RTA-related fatalities in
Austria and the Czech Republic makes it possible to study the response tho the
new law in more detail. To inspect the reaction of fatalities to the traffic
law reform, I first index daily fatalities by the change in the number of cars
and standardize their values by demeaning and dividing by respective standard
deviations of each country.\footnote{Note that daily fatalities are small
  numbers, often taking the value of zero, and there is higher variability of
  fatalities in the Czech Republic relative to Austria. The results do not
  depend on the standardization, however.} Then I run a regression of normalized
fatalities on a constant, a dummy for the Czech Republic, a full set of week
effects, and full set of interactions between the dummy for the Czech Republic
and week effects using daily data ranging from 2005 to 2008.\footnote{I tested
  the equality of pre-July 2006 development in daily fatalities between Austria
  and the Czech Republic with results far from any rejection criteria.
  Pearson's correlation coefficient of pre-July 2006 weekly fatalities in the
  two countries is 0.43 and is highly statistically significant.} 

Panel A of Figure \ref{fig:days} plots the demeaned coefficients on the
interactions capturing the average weekly change in Czech fatalities net of
common shocks.\footnote{More precisely, these are the sums of the coefficients
  on interactions and the coefficient on the dummy for the Czech Republic.} To
improve the readability of the figure, it shows only coefficients for an
18-month window around the reform.  The first day of the reform, July 1, 2006,
is marked by a solid vertical line, and dashed lines mark one week before and
three months after the date. There is no apparent positive or negative trend
during the pre-reform period, which is reassuring.  Neither are there signs of
pre-reform effects. On the contrary, the plot suggests that the law saw its
first effects in the very first week it was enforceable.  Importantly, the
figure corroborates our main finding that the effects of the law were
concentrated in the first three months, as most of the estimates for that period
lie below the zero line.  Specifically, Panel A suggests that the bulk of the
decline in fatalities was concentrated in July and October, the first and third
post-reform month.  Fatalities seem to revert to their pre-reform levels soon
after this initial shock. 

% FIGURE \ref{fig:days} ABOUT HERE
\begin{figure}[t] 
  \caption{Weekly effects of the new traffic law on standardized fatalities (6
  months before--12 months after)}
  \label{fig:days} 
  \makebox[\textwidth]{ 
    \includegraphics[width=17cm, trim=0 0 0 3mm]
    {../tables_figures/daily_fatalities_new.pdf}  
  }
\end{figure}

As an alternative specification, I run a regression of normalized daily
fatalities on a constant, a dummy for the Czech Republic, a full set of
week-of-the-year effects, a third-degree-polynomial trend, and interactions
between the dummy for the Czech Republic and week effects spanning from six
months before to 12 months after the reform.\footnote{Results are the same for
  linear or quadratic trends.} The estimated coefficients on the interaction
terms are plotted in Panel B of Figure \ref{fig:days}. Estimates are less
erratic compared to Panel A, as the restrictive model is likely to filter out
some noise. The main difference between the two plots is that the rebound seems
to be more gradual in Panel B. Otherwise, the interpretation of both figures is
remarkably similar. Daily data thus corroborate our main finding of substantial
immediate effects of the traffic law reform in the Czech Republic and subsequent
rebound toward pre-reform levels.

\section{Why Were the Effects Shortlived?} % {{{1
\label{sec:police}

\subsection{Competing Explanations and Some Theory} % {{{2

The short life of the initial effects of the traffic law reform is not too
surprising. As discussed in the Introduction, this development is consistent
with earlier experiences in other countries.\footnote{See also Table
  \ref{tab:BApapers}.} It, however, is a puzzle, and the empirical literature so
far has not provided much insight into what causes such effects to be
shortlived. I suspect two major factors: expectations and enforcement. 

A radical change in traffic law may be \emph{ex ante} ambiguous with respect to
its effects on expected punishment, to which people may overreact, choosing a
pessimistic scenario and be overly alert
\citep{alary_ambiguity_2012}.\footnote{It is generally better to be
  pessimistic in face of ambiguity, because---\emph{ex post}---one will either
  have been right, or he will be positively surprised.} Then, as the ambiguity
resolves, and people adjust their priors accordingly, their care is relaxed. In
addition, the politicians---both those who are for the reform as well as those
who are against it---the police, and the media may have a tendency exaggerate
the magnitude of the expected effects.\footnote{The Czech traffic law reform
  became enforceable during the holiday season, usually characterised by few
  events in politics, so it may have received even more media attention than
  otherwise.} Also, people may expect, that tougher punishments generate more
incentives for the police to enforce the traffic law.  

However, economic theory suggests the opposite---enforcement levels may have
gone down in the aftermath of the reform. Intuitively, as higher punishments
improve drivers' behavior, enforcement resources may be reallocated to more
valuable uses. \citet{tsebelis_abuse_1989} was the first to study the
relationship between the fine and the effort chosen by the enforcement body. He
discusses a succinct game-theoretical model producing a counter-intuitive
prediction: a change in the fine does not influence the number of committed
infractions, only the probability of capture.\footnote{See also
  \citet{holler_fighting_1993} and \citet{andreozzi_oscilliations_2002} for a
  further discussion of Tsebelis' model.}

In an earlier version of this paper,\footnote{It is available online at
  http://ssrn.com/abstract=1595882 or upon request from the author.} I have
developed a simple model grasping this scenario in a standard constrained
optimization setting. I begin with a straightforward deterrence model, drivers
choose the number of offenses given a monetary constraint. The cost of an
offense is a fine multiplied by the probability that the offender will be
intercepted by the police. In the next step, the police is allowed to optimize
their effort, directly influencing the probability of interception and through
it the number of offenses the drivers choose to commit. The police dislikes
offenses as well as effort. The model predicts that the direct deterrence effect
of a higher fine is followed by relaxed enforcement, which in equilibrium
offsets some of the former.  
  
There are various reasons why this simple model may capture some of the reality.
Resources allocated to traffic police have alternative uses, within the police
and within the public sector in general. It is also plausible that when similar
large-scale changes in the law are adopted, the traffic police may already be
overstretched. If added deterrence resulting from increased punishments leads to
an improvement in road-safety indicators, the levels of enforcement may then be
adjusted downward. This may be an improvement from the social perspective.
Finally, politicians may lose interest in traffic safety, especially if things
improve initially. If they later revert, there may not be much to be done for
them as the law has been passed already.

\subsection{The Data on Police Activity} % {{{2

% FIGURE \ref{fig:police} ABOUT HERE
\begin{figure}[t] 
  \caption{Traffic police manpower and man-hours in enforcement by Czech regions
    in 2006 and 2007
  }
  \label{fig:police} 
  \makebox[\textwidth]{ 
    \includegraphics[width=19cm, trim=0 0 0 5mm]
    {../tables_figures/police_activity}  
  }
\end{figure}

This subsection presents an analysis of a unique dataset parsed from internal
monthly regional-level information on traffic police activity in 2006 and 2007 I
have obtained from the Czech Traffic Police Headquarters. The original source of
the information are the local police directorates, who report to the
Headquarters.

The thrust of the of the findings is captured in Figure \ref{fig:police}, which
plots the development of manpower and man-hours in enforcement as well as the
number of hours of speed gun use by the traffic police across Czech regions. The
number of policemen assigned to enforcement a exhibits general upward pattern
over 2006 and 2007. Despite that, the man-hours worked by traffic policemen were
declining in all regions but one.  The number of hours the police spent behind
speed guns was falling even more rapidly.\footnote{Certainly the enforcement
  infrastructure was improving in recent years. Many static cameras were put in
  place, so that drivers' speed may be measured with higher frequency. However,
  speeding captured by static radars gets recorded, and only a subset proceeds
  through the administrative procedure and possibly results in punishment of the
  driver, with a delay.}

% TABLE \ref{tab:police.activity.measures} ABOUT HERE
\begin{sidewaystable}[p] 
  \footnotesize  
  \makebox[23cm]{
    \begin{threeparttable}
      \caption{Czech Traffic Police Activity in 2006 and 2007}
      \label{tab:police.activity.measures} 
      \input{../tables_figures/tab.police.activity.measures}
      \begin{tablenotes} \scriptsize 
      \item \hspace{-1.8mm} \textsc{Notes}: The outcome variables are in logs.
        The 1\superscript{st} quarter post-reform is the omitted category. All
        specifications include region dummies. Deseasoning is done using a
        coefficient for the second half-year estimated on country-level
        half-yearly data on man-hours in enforcement from 2005 to 2008.
        Huber-White standard errors clustered on regions are in parentheses:
        $\dagger < 0.1$.
      \item \hspace{-1.8mm} \textsc{Sources}: Headquarters of the Police of the
        Czech Republic. 
      \end{tablenotes} 
    \end{threeparttable}
  } 
\end{sidewaystable}

Tables \ref{tab:police.activity.measures} and
\ref{tab:police.alternative.activities} study traffic police activity in more
detail. I simply regress the logs of police activity indicators on
year$\times$quarter dummies, where the second quarter of 2006 is the omitted
category. One may worry that some variability may be driven by seasonal
regularities, an issue hard to deal with properly with two years of data. I use
half-yearly country-level data on man-hours in enforcement ranging from 2005 to
2008 and estimate a coefficient for a second half-year, which I then use to
deseason the monthly data on police activity.

% TABLE \ref{tab:police.alternative.activities} ABOUT HERE
\begin{sidewaystable}[p] 
  \makebox[23cm]{ \footnotesize  
    \begin{threeparttable}   
      \caption{Czech Traffic Police Alternative Activities in 2006 and 2007}
      \label{tab:police.alternative.activities} 
      \input{../tables_figures/tab.pol.alternative.activities}
      \begin{tablenotes}\scriptsize
      \itemNoindent \textsc{Notes}: The outcome variables are in logs. The
        1\superscript{st} quarter post-reform is the omitted category. All
        specifications include region dummies. Deseasoning is done using a
        coefficient for the second half-year estimated on country-level
        half-yearly data on man-hours in enforcement from 2005 to 2008.
        Huber-White standard errors clustered on regions are in parentheses:
        $\dagger < 0.1$.
      \itemNoindent \textsc{Sources}: Headquarters of the Police of the
        Czech
        Republic. 
      \end{tablenotes} 
    \end{threeparttable}
  } 
\end{sidewaystable}

Column (1) of Table \ref{tab:police.activity.measures} suggests that the total
number of policemen remained constant throughout 2006 and 2007, while the number
of policemen in enforcement was slowly increasing as seen in column (2) and
Figure \ref{fig:police}. In other words, there is no sign that fewer policemen
were available for traffic law enforcement after June 2006.\footnote{Also, as
  noted in Section \ref{sec:the-new-law}, the municipal police was newly
  authorized to stop vehicles and fine offenders.} Nonetheless, the total amount
of traffic police man-hours dedicated to enforcement was gradually declining
from the third quarter of 2006 onwards. The declines are statistically
significant at 5 the percent level and the pace was accelerating. Since
man-hours in enforcement are usually smaller in the second half of the year
(although fatalities are higher), the results for deseasoned data places the
beginning of the decline to the first quarter of 2007.  Nonetheless, there were
almost 9\% fewer traffic policemen seen on the streets and roads in the first
quarter following the introduction of the new traffic law compared to the
preceding one, and the decline came close to 24\% by the end of 2007. The decay
is even more dramatic in the case of speed gun hours by the traffic police.  The
number of targeted actions---that is the temporarily increased presence of
traffic police in a specific area with the purpose to increase the number of
checks and the salience of police presence---remained steady through 2006, then
declined in the first half of 2007 and later increased substantially. The
presence of the police in the media decreased in 2007, as it was probably
unusually high in 2006. 

Table \ref{tab:police.alternative.activities} looks at activities of the traffic
police other than direct traffic law enforcement. Specifically, I looked at the
numbers of persons and cars at large found, and the numbers of suspicious items
and persons checked. Although most coefficients are not statistically
significant, there seems to be a general increase in these measures, despite
fewer working hours in enforcement. This suggests that the traffic police may
have given higher priority to general law enforcement activities relative to
pure enforcement of traffic rules.

\subsection{What Does (Not) the Traffic Police Activity Explain?} % {{{2

In sum, the police data reveal that enforcement levels were declining in the
aftermath of the traffic law reform and suggest that traffic police may have put
a higher share of resources into general law enforcement and police work and
away from the direct enforcement of traffic rules. This development is in line
with the outlined theory, however one should be careful before interpreting this
result strictly causally. Only two years of data are available, so we cannot
rule out preexisting trends or factors driving the changes in police data other
than the change in the traffic law.\footnote{A note of caution should be stated
  here: One could be tempted to use the data on traffic police activity as an
  explanatory variable to check whether it explains the development in
  fatalities in the Czech Republic after July 1, 2006.  However, this would
  require that traffic police activity is exogenous to accidents and
  casualties---an unlikely assumption. Doing this would lead to results lacking
  proper interpretation. I therefore present only a descriptive analysis of
  police activity.}

Be it coincidence or not, can changes in police enforcement explain  the
development of fatalities after the change in Czech road traffic law? It can
hardly do so with respect to the sharp short-run decline in fatalities as the
police presence on the roads was apparently lower during the third quarter of
2006, relative to first half of that year. This effect is more likely to be
driven by uncertainty in expectations that were possibly misjudged due to the
high salience of the change as well as the intense media coverage and
long-lasting controversies regarding the DPS. On the other hand, systematically
lower enforcement levels after the law was introduced may play a role in
explaining the absence of long-run effects.

\section{Conclusions} % {{{1 
\label{sec:conclusions} 

This study evaluates the effects of the introduction of a new road traffic law
on RTA-related fatalities in the Czech Republic. The law became effective on
July 1, 2006 and introduced a number of provisions that increased sanctions for
traffic offenses. In particular, fines increased substantially, the traffic
police gained more authority, and a strict demerit point system was introduced.
The law was long disputed as the severity of the increase in punishment was
controversial; the main object of controversy was the demerit point system.  Yet
the reform also had strong supporters, notably the police, as it was expected to
bring a major improvement in drivers' behavior and road traffic safety.

Consistent with the literature, studying the effects of similar changes in
traffic laws in other countries, I find a substantial initial response to the
increase in punishments for traffic offenses. Fatalities were about one-third
lower during the first three months after the law was introduced. From 51 to
204---with point estimate of 119---human lives were saved with 95 percent
certainty in that period. The estimate for the 12 post-reform months is 193
lives saved. However, the estimated effects after the initial three-month period
are generally negative and substantively moderate, but never statistically
significant. The point estimates of the effects going beyond the first 12
months are close to zero, although moderate positive or negative effects cannot
be ruled out. These findings hold across various specifications and are robust
to controlling for the number of cars per inhabitant, GDP per capita, age of
cars, and intensity of freight transport. 

This study extends the set of studies that find strong immediate but little, if
any, sustained effects of tougher punishments for traffic law violations.
Because the expected punishment is what drivers should care about, the key issue
is the development of police resources and effort devoted to the enforcement of
the law. I study a unique detailed dataset parsed from internal reports that
regional traffic police offices provide to the Czech traffic police
headquarters. The data reveals that resources allocated to enforcement were
decaying, namely man-hours and the intensity of the use of speed guns. There are
also indications that traffic police may have shifted attention towards
ancillary activities and more general law enforcement. It is noteworthy in this
context that traffic intensity, measured by number of cars, kilometers driven,
or intensity of freight transport, was increasing substantially during the
period under study. Traffic-intensity-adjusted enforcement levels may have
decayed even faster.

A continuous decline in traffic law enforcement cannot, however, explain the
short-run pattern of the reaction to the legal reform. It is plausible to
hypothesize that people overestimated the effects of the law on expected
punishment, which were ambiguous in the first place. This may have led them to
be overly careful during first post-reform weeks. As the uncertainty resolved,
people corrected their priors and fatalities rebounded. A proper evaluation of
the importance of this mechanism is a task for future research.

The paper does not imply that the law was not an improvement. To the contrary,
it created a legal environment that is closer to the standards found in the rest
of Europe. By introducing the demerit point system, sanctions became more
independent of income (at least formally), thus providing added deterrence and
incapacitation of drivers that may not perceive fines as biting enough---there
were as many as 34,000 drivers whose driver's licenses were revoked through DPS
as of December 2011. Nevertheless, the effective sanctions that offenders care
about are a function of enforcement. The expectation that a nominal increase in
penalties fix the problem may take enforcement out of the focus and the police
and/or politicians may have incentives to reduce enforcement, especially if
things seem to go well initially. This this is consistent with the data on
police activity and may help explain the absence of any long-run effects.

\bibliography{DPS.bib}

% \newpage

% \section*{Figures} % {{{1

\newpage

% \section*{Tables} % {{{1

\FloatBarrier
\appendix

\section*{Appendix} % {{{1
\setcounter{table}{0}
\renewcommand\thetable{A.\arabic{table}}

\begin{sidewaystable}
  \renewcommand{\arraystretch}{1.2}
  \makebox[23cm]{
    \footnotesize 
    \begin{threeparttable} 
      \caption{Before-After Studies from Other Countries} 
      \label{tab:BApapers}
      \input{../tables_figures/tab-BApapers.csv}
      \begin{tablenotes} \scriptsize 
      \item \hspace{-1.8mm} \textsc{Notes}: Effects beyond the initial 6 months
        are marked as N.A. if the after-period does not extend beyond 6 months,
        or if it is 12 months long but the study does not disaggregate analysis
        into shorter periods.
      \end{tablenotes}
    \end{threeparttable}
  }
\end{sidewaystable}

\begin{table}[] 
  \renewcommand{\arraystretch}{1.1}
  \centering
  \caption{Overview of Novel Provisions in the 2006 Traffic Law}  
  \label{tab:provisions}  	 
  \footnotesize  
  \input{../tables_figures/tab-newrules}	
\end{table} 

% TABLE \ref{tab:summary2} ABOUT HERE
\begin{table}[] 
  \scriptsize 
  \makebox[\textwidth] {
    \begin{threeparttable} 
      \caption{Summary of Transport and Economic Statistics (Means of 2004/2005
        and 2007/2008)
      }       
      \label{tab:summary2}   
      \begin{tabular}
        {
          l c *{2}{m{1.4cm}<{\centering}} *{1}{m{1.3cm}<{\centering}}
          *{1}{m{1cm}<{\centering}} *{1}{m{1.3cm}<{\centering}}
          *{1}{m{1.1cm}<{\centering}}*{1}{m{1.5cm}<{\centering}}
        }
        \toprule 
        \input{../tables_figures/tab-summary2a.csv}
      \end{tabular} 
      \begin{tablenotes} \scriptsize 
      \item \hspace{-1.8mm} \textsc{Notes}: The values of the age of cars,
        freight transport, and highways refer to all of Germany.
      \item \hspace{-1.8mm} \textsc{Sources}: Eurostat.
      \end{tablenotes} 
    \end{threeparttable} 
  }
\end{table}

\end{document}

